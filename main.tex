\documentclass[10pt,aspectratio=169]{beamer}

\usepackage{spbu}

\usepackage[T2A]{fontenc}
\usepackage[utf8]{inputenc}
\usepackage[english,russian]{babel}

\title{Шаблон презентации для СПбГУ}
\subtitle{\LaTeX/Beamer}
\author{Фамилия Имя Отчество}


\begin{document}

    \maketitle

    \section{Введение}

    \begin{frame}{Пакет Beamer для подготовки презентации}
        \begin{itemize}
            \item Предполагается, что вы умеете пользоваться \LaTeX, если это не так, вы можете начать
                \href{https://www.latex-project.org/}{ознакомление с ним здесь}.
            \item Beamer --- один из самых популярных и мощных пакетов для подготовки презентаций в \LaTeX.
            \item У пакета Beamer есть подробное
                \href{http://www.ctan.org/tex-archive/macros/latex/contrib/beamer/doc/beameruserguide.pdf}{руководство пользователя}.
            \item Данная презентация представляет из себя шаблон/пример, подготовленный с использованием пакета Beamer с фирменной символикой СПбГУ.
        \end{itemize}
    \end{frame}

    \section{Редактирование}

    \begin{frame}[fragile]{Использование Beamer}
        Для создания простейшего документа с использованием пакета Beamer достаточно воспользоваться следующим кодом:
        \begin{block}{Минимальный документ}
            \begin{lstlisting}[language=TeX]
\documentclass{beamer}
\begin{document}
    \begin{frame}{Hello, world!}
    \end {frame}
\end{document}
            \end{lstlisting}
        \end{block}
    \end{frame}

    \begin{frame}[fragile]{Титульный слайд}
        Чтобы отобразить информацию на титульном слайде, необходимо вызвать несколько команд в преамбуле:
        \begin{block}{Команды для вывода информации на титульный слайд}
            \begin{lstlisting}[language=TeX]
\title{Title}
\subtitle{Subtitle}
\author{First Author, Second Author}
\date{Defaults to today's}
            \end{lstlisting}
        \end{block}
        Затем, после команды \verb|\begin{document}|, необходимо написать команду \verb|\maketitle|.
    \end{frame}

    \begin{frame}[fragile]{Создание слайда с информацией}
        \framesubtitle{Руководство}
        Для создания простого слайда, необходимо воспользоваться следующим кодом:
        \begin{block}{Код для создания слайда}
            \begin{lstlisting}[language=TeX]
\begin{frame}
    \frametitle{Slide example}
    \framesubtitle{Result}
\end{frame}\end{lstlisting}
        \end{block}
        Результат представлен на следующем слайде.

        p.s. Для возможности использования русского текста в листинге, можно воспользоваться пакетом \href{https://ctan.org/pkg/listingsutf8?lang=en}{listingsutf8}.
    \end{frame}

    \begin{frame}
        \frametitle{Slide example}
        \framesubtitle{Result}
    \end{frame}

    \begin{frame}[fragile]{Добавление изображений}
        \begin{columns}
            \begin{column}{0.7\textwidth}
                Добавление изображений работает как в стандартном \LaTeX:
                \begin{block}{Добавление изображения}
                    \begin{lstlisting}[language=TeX]
\usepackage{graphicx}
% ...
\includegraphics
[width=\textwidth]{images/default}
                    \end{lstlisting}
                \end{block}
            \end{column}
            \begin{column}{0.3\textwidth}
                \includegraphics[width=\textwidth]{source/logo-name}\\
            \end{column}
        \end{columns}
    \end{frame}

    \begin{frame}[fragile]{Добавление формул}
        Формулы добавляются также как и в стандартном \LaTeX:
        \begin{itemize}
            \item Строчные формулы добавляются с использованием символа доллар (\$), например\\
            \verb|$\Delta =\nabla^{2}$|: $\Delta =\nabla^{2}$.
            \item Вынесенные формулы добавляются, например с помощью двух символов доллара, либо с использованием окружения
            \verb|equation|, любо любым другим предпочтительным для вас способом.
        \end{itemize}
    \end{frame}

    \begin{frame}[fragile]{Добавление формул}
        \framesubtitle{Пример вынесенной формулы}

        \begin{block}{Добавление вынесенной формулы}
            \begin{lstlisting}[language=TeX]
\begin{equation}
    \frac{\partial u}{\partial t}
    -a^2\left(\frac{\partial^2 u}{\partial x^2_1}
    +\frac{\partial^2 u}{\partial x^2_2}
    +\ldots+\frac{\partial^2 u}{\partial x^2_n}\right)=f\left(x,t\right).
    \label{eq:eq1}
\end{equation}
            \end{lstlisting}
        \end{block}

        Результат:
        \begin{equation}
            \frac{\partial u}{\partial t}
            -a^2\left(\frac{\partial^2 u}{\partial x^2_1}
            +\frac{\partial^2 u}{\partial x^2_2}
            +\ldots+\frac{\partial^2 u}{\partial x^2_n}\right)=f\left(x,t\right).
            \label{eq:eq1}
        \end{equation}
        Ссылка на формулу (\ref{eq:eq1}).
    \end{frame}

    \begin{frame}[fragile]{Разделение на столбцы}
        Разделение информации на столбцы осуществляется следующим способом
        \begin{block}{Разделение информации на столбцы}
            \begin{lstlisting}[language=TeX]
\begin{columns}
    \begin{column}{0.6\textwidth}
        This is the first column
    \end{column}
    \begin{column}{0.3\textwidth}
        And this the second
    \end{column}
    % There could be more!
\end{columns}
            \end{lstlisting}
        \end{block}

        Далее представлен результат:
        \begin{columns}
            \begin{column}{0.6\textwidth}
                First column.
            \end{column}
            \begin{column}{0.3\textwidth}
                Second column.
            \end{column}
        \end{columns}
    \end{frame}

    \begin{frame}[fragile]
        \frametitle{Шрифты}
        Для представления различной информации, уместно использовать соответствующие ей шрифты:
        \begin{itemize}
            \item {\verb|\textrm{}|: \textrm{стандартный шрифт;}}
            \item {\verb|\textsf{}|: \textsf{шрифт семейства sans-serif;}}
            \item {\verb|\texttt{}|: \texttt{моноширинный шрифт.}}
        \end{itemize}
    \end{frame}

    \section{Заключение}

    \begin{frame}
        \frametitle{Заключение}
        \begin{itemize}
            \item Автор будет рад, вашему вкладу в репозиторий на \href{github.com}{GitHub}.
            \item Вопросы, пожелания предложения можно отправлять на
            \href{mailto:gleb.alcybeev@gmail.com}{электронную почту}.
        \end{itemize}
    \end{frame}

    \backmatter

\end{document}
